%% Generated by Sphinx.
\def\sphinxdocclass{jupyterBook}
\documentclass[a4paper,12pt,english]{jupyterBook}
\ifdefined\pdfpxdimen
   \let\sphinxpxdimen\pdfpxdimen\else\newdimen\sphinxpxdimen
\fi \sphinxpxdimen=.75bp\relax
\ifdefined\pdfimageresolution
    \pdfimageresolution= \numexpr \dimexpr1in\relax/\sphinxpxdimen\relax
\fi
%% let collapsible pdf bookmarks panel have high depth per default
\PassOptionsToPackage{bookmarksdepth=5}{hyperref}
%% turn off hyperref patch of \index as sphinx.xdy xindy module takes care of
%% suitable \hyperpage mark-up, working around hyperref-xindy incompatibility
\PassOptionsToPackage{hyperindex=false}{hyperref}
%% memoir class requires extra handling
\makeatletter\@ifclassloaded{memoir}
{\ifdefined\memhyperindexfalse\memhyperindexfalse\fi}{}\makeatother

\PassOptionsToPackage{warn}{textcomp}

\catcode`^^^^00a0\active\protected\def^^^^00a0{\leavevmode\nobreak\ }
\usepackage{cmap}
\usepackage{fontspec}
\defaultfontfeatures[\rmfamily,\sffamily,\ttfamily]{}
\usepackage{amsmath,amssymb,amstext}
\usepackage{polyglossia}
\setmainlanguage{english}



\setmainfont{FreeSerif}[
  Extension      = .otf,
  UprightFont    = *,
  ItalicFont     = *Italic,
  BoldFont       = *Bold,
  BoldItalicFont = *BoldItalic
]
\setsansfont{FreeSans}[
  Extension      = .otf,
  UprightFont    = *,
  ItalicFont     = *Oblique,
  BoldFont       = *Bold,
  BoldItalicFont = *BoldOblique,
]
\setmonofont{FreeMono}[
  Extension      = .otf,
  UprightFont    = *,
  ItalicFont     = *Oblique,
  BoldFont       = *Bold,
  BoldItalicFont = *BoldOblique,
]



\usepackage[Bjarne]{fncychap}
\usepackage[,numfigreset=1,mathnumfig]{sphinx}

\fvset{fontsize=\small}
\usepackage{geometry}


% Include hyperref last.
\usepackage{hyperref}
% Fix anchor placement for figures with captions.
\usepackage{hypcap}% it must be loaded after hyperref.
% Set up styles of URL: it should be placed after hyperref.
\urlstyle{same}


\usepackage{sphinxmessages}



        % Start of preamble defined in sphinx-jupyterbook-latex %
         \usepackage[Latin,Greek]{ucharclasses}
        \usepackage{unicode-math}
        % fixing title of the toc
        \addto\captionsenglish{\renewcommand{\contentsname}{Contents}}
        \hypersetup{
            pdfencoding=auto,
            psdextra
        }
        % End of preamble defined in sphinx-jupyterbook-latex %
        

\title{Wykonywalne książki}
\date{Mar 10, 2022}
\release{}
\author{Wojciech Broniowski}
\newcommand{\sphinxlogo}{\vbox{}}
\renewcommand{\releasename}{}
\makeindex
\begin{document}

\pagestyle{empty}
\sphinxmaketitle
\pagestyle{plain}
\sphinxtableofcontents
\pagestyle{normal}
\phantomsection\label{\detokenize{docs/index::doc}}




\sphinxAtStartPar
\sphinxhref{https://www.ujk.edu.pl/~broniows}{\sphinxstylestrong{Wojciech Broniowski}}





\sphinxAtStartPar
Przedstawię, jak można w łatwy sposób sporządzić książkę (Jupyter Book), zawierającą wykonywalne programy w Pythonie. Progamy nie wymagają żadnej instalacji, a student może je dowolnie modyfikować, zmieniać parametry itp. Wielu z nas ma wykłady dot. programowania w Pyhonie, gdzie wykonywalna książka jest świetnym sposobem uporządkowania notatek i dania studentom znakomitej pomocy dydaktycznej. Opowiem też, jak opublikować książkę w Google Books.

\begin{sphinxadmonition}{note}{Linki}
\begin{itemize}
\item {} 
\sphinxAtStartPar
Jupyter Book:
\sphinxurl{https://bronwojtek.github.io/sem/docs/index.html}

\item {} 
\sphinxAtStartPar
pdf seminarum: \sphinxurl{https://github.com/bronwojtek/sem/blob/master/ExeBook.pdf}

\end{itemize}
\end{sphinxadmonition}

\begin{sphinxadmonition}{note}{\protect\(~\protect\)}

\sphinxAtStartPar
Built with \sphinxhref{https://beta.jupyterbook.org/intro.html}{Jupyter Book
2.0} tool set, as part of the
\sphinxhref{https://ebp.jupyterbook.org/en/latest/}{ExecutableBookProject}.
\end{sphinxadmonition}


\chapter{Przykładowy rozdział}
\label{\detokenize{docs/rozdz_1:przykladowy-rozdzial}}\label{\detokenize{docs/rozdz_1::doc}}
\sphinxAtStartPar
Komórki trojakiego rodzaju


\section{Jakiś podrozdział}
\label{\detokenize{docs/rozdz_1:jakis-podrozdzial}}
\sphinxAtStartPar
Rozumie latex:

\sphinxAtStartPar
\(w_0+w_1 x_1 + w_2 x_2 > 0\).

\begin{sphinxuseclass}{cell}\begin{sphinxVerbatimInput}

\begin{sphinxuseclass}{cell_input}
\begin{sphinxVerbatim}[commandchars=\\\{\}]
\PYG{k}{def} \PYG{n+nf}{sq}\PYG{p}{(}\PYG{n}{x}\PYG{p}{)}\PYG{p}{:}
    \PYG{k}{return} \PYG{n}{x}\PYG{o}{*}\PYG{n}{x}  \PYG{c+c1}{\PYGZsh{} square}
\end{sphinxVerbatim}

\end{sphinxuseclass}\end{sphinxVerbatimInput}

\end{sphinxuseclass}
\begin{sphinxuseclass}{cell}\begin{sphinxVerbatimInput}

\begin{sphinxuseclass}{cell_input}
\begin{sphinxVerbatim}[commandchars=\\\{\}]
\PYG{c+c1}{\PYGZsh{} sigmoid }
\PYG{k}{def} \PYG{n+nf}{sig\PYGZus{}T}\PYG{p}{(}\PYG{n}{s}\PYG{p}{,}\PYG{n}{T}\PYG{p}{)}\PYG{p}{:}
    \PYG{k}{return} \PYG{l+m+mi}{1}\PYG{o}{/}\PYG{p}{(}\PYG{l+m+mi}{1}\PYG{o}{+}\PYG{n}{np}\PYG{o}{.}\PYG{n}{exp}\PYG{p}{(}\PYG{o}{\PYGZhy{}}\PYG{n}{s}\PYG{o}{/}\PYG{n}{T}\PYG{p}{)}\PYG{p}{)}
\end{sphinxVerbatim}

\end{sphinxuseclass}\end{sphinxVerbatimInput}

\end{sphinxuseclass}
\begin{sphinxuseclass}{cell}
\begin{sphinxuseclass}{tag_remove-input}\begin{sphinxVerbatimOutput}

\begin{sphinxuseclass}{cell_output}
\noindent\sphinxincludegraphics{{rozdz_1_8_0}.png}

\end{sphinxuseclass}\end{sphinxVerbatimOutput}

\end{sphinxuseclass}
\end{sphinxuseclass}

\chapter{Rozdział 2}
\label{\detokenize{docs/rozdz_2:rozdzial-2}}\label{\detokenize{docs/rozdz_2::doc}}
\sphinxAtStartPar
Można chować nieistotne komórki

\begin{sphinxuseclass}{cell}\begin{sphinxVerbatimInput}

\begin{sphinxuseclass}{cell_input}
\begin{sphinxVerbatim}[commandchars=\\\{\}]
\PYG{k}{def} \PYG{n+nf}{cube}\PYG{p}{(}\PYG{n}{x}\PYG{p}{)}\PYG{p}{:}
    \PYG{k}{return} \PYG{n}{x}\PYG{o}{*}\PYG{o}{*}\PYG{l+m+mi}{3}  \PYG{c+c1}{\PYGZsh{} cube}
\end{sphinxVerbatim}

\end{sphinxuseclass}\end{sphinxVerbatimInput}

\end{sphinxuseclass}
\begin{sphinxuseclass}{cell}\begin{sphinxVerbatimInput}

\begin{sphinxuseclass}{cell_input}
\begin{sphinxVerbatim}[commandchars=\\\{\}]
\PYG{n}{y}\PYG{o}{=}\PYG{l+m+mi}{2}
\PYG{n+nb}{print}\PYG{p}{(}\PYG{n}{y}\PYG{p}{,} \PYG{l+s+s1}{\PYGZsq{}}\PYG{l+s+s1}{ }\PYG{l+s+s1}{\PYGZsq{}}\PYG{p}{,} \PYG{n}{cube}\PYG{p}{(}\PYG{n}{y}\PYG{p}{)}\PYG{p}{)}
\end{sphinxVerbatim}

\end{sphinxuseclass}\end{sphinxVerbatimInput}
\begin{sphinxVerbatimOutput}

\begin{sphinxuseclass}{cell_output}
\begin{sphinxVerbatim}[commandchars=\\\{\}]
2   8
\end{sphinxVerbatim}

\end{sphinxuseclass}\end{sphinxVerbatimOutput}

\end{sphinxuseclass}






\renewcommand{\indexname}{Index}
\printindex
\end{document}